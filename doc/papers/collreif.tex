\documentclass{llncs}
\usepackage{makeidx}  % allows for indexgeneration
\usepackage[pdfpagescrop={92 112 523 778},a4paper=false,
            pdfborder={0 0 0}]{hyperref}
\emergencystretch=8pt
%
\begin{document}
\mainmatter              % start of the contributions
%
\title{Model for Collaborative Decision Making Based on RDF Reification}
\toctitle{Model for Collaborative Decision Making Based on RDF Reification}
\titlerunning{Collaboration and RDF Reification}
%
\author{Dmitry Borodaenko}
\authorrunning{Dmitry Borodaenko}   % abbreviated author list (for running head)
%%%% modified list of authors for the TOC (add the affiliations)
\tocauthor{Dmitry Borodaenko}
%
\institute{\email{angdraug@debian.org}}

\maketitle              % typeset the title of the contribution

\begin{abstract}
This paper presents a novel approach to online collaboration on the Web,
intended as technical means to make collective decisions in situations when
consensus fails. It is proposed that participants of the process are allowed
to create statements about site resources and, by the means of RDF
reification, to assert personal approval of such statements. Arbitrary
algorithms may then be used to determine validity of a statement in a given
context from the set of approval statements by different participants. The
paper goes on to discuss applicability of the proposed approach in the areas
of open-source development and independent media, and describes its
implementation in the Samizdat open publishing and collaboration system.
\end{abstract}


\section{Introduction}

Extensive growth of Internet over the last decades introduced a new form of
human collaboration: online communities. Availability of cheap digital
communication media has made it possible to form large distributed projects,
bringing together participants who would be otherwise unable to cooperate.

As more and more projects go online and spread across the globe, it becomes
apparent that new opportunities in remote cooperation also bring forth new
challenges. As observed by Steven Talbott\cite{fdnc}, technogical means do not
provide a full substitute for a real person-to-person relations, ``technology
is not a community''. A well-known example of this is the fact that it is
vital for an online communty to augment indirect and impersonal digital
communications with live meetings. However, even regular live meetings do not
solve all of the remote cooperation problems as they are limited in time and
scope, and thus can't happen often enough nor include all of the interested
parties into communication. In particular, one of the problems of online
communities that is begging for a new and better technical solution is
decision making and dispute resolution.

While it is most common that online communities are formed by volunteers,
their forms of governance are not necessarily democratic and vary widely, from
primitive single-person leadership and meritocracy in less formal technical
projects to consensus and majority voting in more complicated situations.

Usually, decision making in online volunteer projects is carried out via
traditional communication means, such as IRC channels, mailing lists,
newsgroups, etc., with rare exceptions such as the Debian project which
employs its own Devotee voting system based on PGP authentication and Concorde
vote counting\cite{debian-constitution}, and the Wikipedia project which
relies on a Wiki collaborative publishing system and enforces consensus among
its contributors. The scale and the level of quality achieved by the latter
two projects demonstrates that formalized collaboration process is as
important for volunteer projects as elsewhere: while sufficient to determine
rough consensus, traditional communications require participants to come up
with informal means of dispute resolution, making the whole process overly
dependent on interpersonal attitudes and communicative skills within group.

It is not to say that Debian or Wikipedia processes are perfect and need not
be improved. The strict consensus required by the Wikipedia Editors Policy
discourages dissenting minority from participation, while full-scale voting
system like Debian Devotee can't be used for every minor day-to-day decision
because of the high overhead involved and the limits imposed by the ballot
form.

This paper describes how RDF statement approval based on reification can be
applied to the problem of online decision making in diverse and politically
intensive distributed projects, and proposes a generic semantic model which
can be used in a wide range of applications involving online collaboration.
The proposed model is implemented in the Samizdat open-publishing and
collaboration engine, described later in the paper.


\section{Collaboration Model}

The collaboration model implemented by Samizdat evolves around the concept of
\emph{open editing}\cite{opened}, which includes the processes of publishing,
structuring, and filtering online content. ``Open'' part of open editing
implies that the collaboration process is visible to all participants, and
roles of readers and editors are available equally to everyone.
\emph{Publishing} involves posting new documents, comments, and revised
documents. \emph{Structuring} involves categorization and appraisal of
publications and other actions of fellow participants. \emph{Filtering}
process is intended to reduce information flow to a comprehensible level by
presenting a user with resources of highest quality and relevance. Each of
these processes requires a fair amount of decision making to be done, which
means that its effectiveness can be greatly improved by automating some
aspects of the decision making procedure.


\section{Collective Statement Approval}
%
\subsection{Focus-Centered Site Structure}

In the proposed collaboration model, RDF statements are used as a generic
mechanism for structuring site content. While it is possible to make any kinds
of statements about site resources, the most important kind of statement is
the one that relates a resource to a so-called ``focus''\cite{concepts}.
\emph{Focus} is a kind of resource that, when related by an RDF statement to
other resources, allows to group similar resources together and to evaluate
resources against different criteria. In some sense, all activities of project
members are represented as relations between resources and focuses.

Dynamically grouping resources around different focuses allows project members
to concentrate on the resources that are most relevant to their area of
interests and provide best quality. Use of RDF for site structure description
makes it possible to store and exchange filters for site resource selection in
the form of RDF queries, thus allowing participants to share their preferences
and ensuring interoperability with RDF-aware agents.

Since any resource can be used as a focus, it is possible that project members
define their own focuses, and relate focuses one to another. In a sufficiently
large and intensive project, this feature should help site structure to evolve
in accordance with usage patterns of different groups of users.

\subsection{RDF Reification}

RDF reification provides a mechanism for describing RDF statements. As defined
in ``RDF Semantics''\cite{rdf-mt}, assertion of reification of RDF statement
means that a document exists containing a triple token instantiating the
statement. The reified triple is a resource which can be described in the same
way as any other resource. It is important to note that there can be several
triple tokens with the same subject, object, and predicate, and, according to
RDF reification semantics, such tokens should be treated as separate
resources, possibly with different composition or provenance information
attached to each.

\subsection{Proposition and Vote}

In the proposed model, all statements are reified, and may be voted upon by
project members. To distinguish statements with attached votes, they are
called ``propositions''. \emph{Proposition} is a subclass of RDF statement
which can be approved or disapproved by votes of project members. Accordingly,
\emph{vote} is a record of vote cast in favor or against particular
proposition by particular member, and \emph{rating} is a denotation of
approval of the proposition as determined from individual votes.

Exact mechanism of rating calculation can be determined by each site, or even
each user, individually, according to average value of votes cast, level of
trust existing between the user and particular voters, absolute number of
votes cast, etc. Since individual votes are recorded in RDF and are available
for later extraction, rating can be calculated at any time using any formula
that suits the end user best. Some users may choose to share their view of the
site resources, and publish their filters in the form of RDF queries.

Default rating system in Samizdat lets voter select from ratings ``$-2$''
(no), ``$-1$'' (not likely), ``$0$'' (uncertain), ``$1$'' (likely), ``$2$''
(yes). Total rating of proposition is equal to the average value of all votes
cast for the proposition; resources with rating below ``$-1$'' are hidden from
view.


\section{Target Applications and Use Cases}
%
\subsection{Open Publishing}

While it is vital for any project to come up with fair and predictable methods
of decision making, it's hard to find a more typical example than the
Indymedia network, international open publishing project with the aim of
providing the public with unbiased news source\cite{openpub}. Since the main
focus of Indymedia is politics, and since it is explicitly open for everyone,
independent media centers are used by people from all parts of political
spectrum, and often become a place of heated debate, or even target of flood
attacks.

This conflict between fairness and political bias, as well as sheer amount of
information flowing through the news network, creates a need for a more
flexible categorization and filtering system that would take the burden and
responsibility of moderation off from site administrators. The issue of
developing an open editing system was raised by Indymedia project participants
in January 2002, but, to date, implementations of this concept are not ready
for production use. The Active2 project\cite{active2} which has set forth to
fulfil that role is still in the alpha stage of the development, and, unlike
Samizdat, limits its use of RDF to describing its resources with Dublin Core
meta-data.

Implementation of an open editing system was one of the initial goals of the
Samizdat project\cite{oscom3}, and deployment of the Samizdat engine by an
independent media center would become a deciding trial of vitality of the
proposed collaboration model in a real-world environment.

\subsection{Documentation Development}

Complexity level of modern computer systems makes it impossible to develop and
operate them without extensive user and developer manuals which document
intended behaviour of a system and describe solutions to typical user
problems. Ultimately, such manuals reflect collective knowledge about a
system, and may require input from many different people with different
perspectives. On the other hand, in order to be useful to different people,
documentation should be well-structured and easy to navigate.

The most popular solution for collaborative documentation development to date
is \emph{Wiki}, a combination of very simple hypertext markup and ability to
edit documents within an HTML form. Such simplicity makes Wiki easy to use,
but in the same time limits its applicability to large bodies of
documentation. Due to being limited to basic hypertext without categorization
and filtering capabilities, Wiki sites require huge amount of manual editing
done by trusted maintainers in order to keep the site structure from falling
behind a growing amount of available information, and to protect it from
vandals. Although there are successful examples of large Wiki sites (most
prominent being the Wikipedia project), Wiki does not provide sufficient
infrastructure for development and maintainance of complex technical
documentation.

Combination of the Wiki approach with RDF metadata, along with implementation
of the proposed collaborative decision making model for determination of
documentation structure, would allow to make significant progress in the
adoption of the open-source software which is often suffering from a lack of
comprehensive and up-to-date documentation.

\subsection{Bug Tracking}

Bug-tracking tools have grown to become essential component of any software
development process. However, despite wide adoption, bug-tracking software has
not yet reached maturity: interoperability between different tools is missing;
incompatible issue classifications and work flows complicate status
syncronization between companies collaborating on a single project; lack of
integration with time-management, document management, version control and
other kinds of applications increases amount of routine work done by project
manager.

On the other hand, development of integrated project management systems shows
that the most important problem in project management automation is
convergence of information from all sources in a single focal point. For such
convergence to become possible, unified process flow model, based on open
standards such as RDF, should be adopted across all information sources, from
source code version control to developer forums. Since strict provenance
tracking is a key requirement for such model, the proposed reification-based
approach may be employed to satisfy it.


\section{Samizdat Engine}
%
\subsection{Project Status}

Samizdat engine is implemented in the Ruby programming language and relies on
the PostgreSQL database management system for RDF storage. Other programs
required for Samizdat deployment are Ruby/Postgres, Ruby/DBI, and YAML4R
libraries for Ruby, and Apache web server with mod\_ruby module. Samizdat is
free software and does not require any non-free software to
run\cite{impl-report}.

Samizdat project development started in December 2002, first public release
was announced in June 2003. As of the second beta version 0.5.1, released in
March 2004, Samizdat provided basic set of open publishing functionality,
including registering site members, publishing and replying to messages,
uploading multimedia messages, voting on relation of site focuses to
resources, creating and managing new focuses, hand-editing or using GUI for
constructing and publishing Squish queries that can be used to search and
filter site resources. Next major release 0.6.0 is expected to add
collaborative documentation development functionality.

\subsection{Samizdat Schema}

Core representation of Samizdat content is RDF. Any new resource published on
Samizdat site is automatically assigned a unique numberic ID, which, when
appended to the base site URL, forms resource URIref. This ID may be accessed
via {\tt id} property. Publication time stamp is recorded in {\tt dc:date}
property (here and below, ``{\tt dc:}'' prefix refers to the Dublin Core
namespace):

\begin{verbatim}
:id
        rdfs:domain rdfs:Resource .

dc:date
        rdfs:domain rdfs:Resource .
\end{verbatim}

{\tt Member} is a registered user of a Samizdat site (synonyms: poster,
visitor, reader, author, creator). Members can post messages, create focuses,
relate messages to focuses, vote on relations, view messages, use and publish
filters based on relations between messages and focuses.

\begin{verbatim}
:Member
        rdfs:subClassOf rdfs:Resource .

:login
        rdfs:domain :Member ;
        rdfs:range rdfs:Literal .
\end{verbatim}

Resources are related to focuses with {\tt dc:relation} property:

\begin{verbatim}
:Focus
        rdfs:subClassOf rdfs:Resource .

dc:relation
        rdfs:domain rdfs:Resource ;
        rdfs:range :Focus .
\end{verbatim}

{\tt Proposition} is an RDF statement with {\tt rating} property. Value of
{\tt rating} is calculated from {\tt voteRating} values of individual {\tt
Vote} resources attached to this proposition via {\tt voteProposition}
property:

\begin{verbatim}
:Proposition
        rdfs:subClassOf rdf:Statement .

:rating
        rdfs:domain :Proposition ;
        rdfs:range rdfs:Literal .

:Vote
        rdfs:subClassOf rdfs:Resource .

:voteProposition
        rdfs:domain :Vote ;
        rdfs:range :Proposition .

:voteMember
        rdfs:domain :Vote ;
        rdfs:range :Member .

:voteRating
        rdfs:domain :Vote ;
        rdfs:range rdfs:Literal .
\end{verbatim}

Parts of Samizdat schema that are not relevant to the discussed collective
decision making model, such as discussion threads, version control, and
aggregate messages, were omitted. Full Samizdat schema in N3 notation can be
found in Samizdat source code package.

\subsection{RDF Storage Implementation}

To address scalability concerns, Samizdat extends traditional relational
representation of RDF as a table of \{subject, object, predicate\} triples
with a unique RDF-to-relational query translation technology. Most highly used
RDF properties of Samizdat schema are mapped into fields of \emph{internal
resource tables} corresponding to resource classes, with id of the record
referencing to the {\tt Resource} table; all other properties are recorded as
triples in the {\tt Statement} table. Detailed explanation of the
RDF-to-relational mapping can be found in ``Samizdat RDF
Storage''\cite{rdf-storage} document.

To demonstrate usage of the Samizdat RDF schema described earlier in this
section, the exerpt of Ruby code responsible for individual vote rating
assignment is quoted below.

\begin{verbatim}
def rating=(value)
    value = Focus.validate_rating(value)
    if value then
        rdf.assert %{
UPDATE ?rating = '#{value}'
WHERE (rdf::subject ?stmt #{resource.id})
      (rdf::predicate ?stmt dc::relation)
      (rdf::object ?stmt #{@id})
      (s::voteProposition ?vote ?stmt)
      (s::voteMember ?vote #{session.id})
      (s::voteRating ?vote ?rating)
USING PRESET NS}
        @rating = nil   # invalidate rating cache
    end
end
\end{verbatim}

In this attribute assignment method of {\tt Focus} class, RDF assertion is
recorded in extended Squish syntax and populated with variables storing the
rating {\tt value}, resource identifier {\tt resource.id}, focus identifier
{\tt @id}, and identifier of registered member {\tt session.id}. When the
Samizdat RDF storage layer updates {\tt Vote.voteRating}, average value of
corresponding {\tt Proposition.rating} is recalculated by a stored procedure.


\section{Conclusions}

Initially started as an RDF-based open-publishing engine, Samizdat project
opens a new approach to online collaboration in general. Proposed model of
collective statement approval via RDF reification is applicable in a large
range of problem domains, including documentation development and bug
tracking.

Implementation of the proposed model in the Samizdat engine proves viability
of RDF not only as a metadata interchange format, but also as a data model
that may be employed by software architects in innovative ways. Key role
played by RDF reification in the described model shows that this comparatively
obscure part of RDF standard deserves broader mindshare among Semantic Web
developers.


% ---- Bibliography ----
%
\begin{thebibliography}{19}
%
\bibitem {openpub}
Arnison, Matthew:
Open publishing is the same as free software, 2002\\
http://www.cat.org.au/maffew/cat/openpub.html

\bibitem {concepts}
Borodaenko, Dmitry:
Samizdat Concepts, December 2002\\
http://savannah.nongnu.org/cgi-bin/viewcvs/samizdat/samizdat/doc/\\
concepts.txt

\bibitem {rdf-storage}
Borodaenko, Dmitry:
Samizdat RDF Storage, December 2002\\
http://savannah.nongnu.org/cgi-bin/viewcvs/samizdat/samizdat/doc/\\
rdf-storage.txt

\bibitem {oscom3}
Borodaenko, Dmitry:
Samizdat --- RDF model for an open publishing and cooperation engine. Third
International OSCOM Conference, Berkman Center for Internet and Society,
Harvard Law School, May 2003\\
http://slideml.bitflux.ch/files/slidesets/503/title.html

\bibitem {impl-report}
Borodaenko, Dmitry:
Samizdat RDF Implementation Report, September 2003\\
http://lists.w3.org/Archives/Public/www-rdf-interest/2003Sep/0043.html

\bibitem {debian-constitution}
Debian Constitution. Debian Project, 1999\\
http://www.debian.org/devel/constitution

\bibitem {rdf-mt}
Hayes, Patrick:
RDF Semantics. W3C, February 2004\\
http://www.w3.org/TR/rdf-mt

\bibitem {opened}
Jay, Dru:
Three Proposals for Open Publishing --- Towards a transparent, collaborative
editorial framework, 2002\\
http://dru.ca/imc/open\_pub.html

\bibitem {fdnc}
Talbott, Stephen L.:
The Future Does Not Compute. O'Reilly \& Associates, 1995\\
http://www.oreilly.com/\homedir{}stevet/fdnc/

\bibitem {active2}
Warren, Mike:
Active2 Design. Indymedia, 2002.\\
http://docs.indymedia.org/view/Devel/DesignDocument

\end{thebibliography}
\end{document}
